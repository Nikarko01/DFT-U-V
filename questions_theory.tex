\section{Questions}

\paragraph{Q1} If $E_U$ is the sum on ever atom of $U^I$ then why in my MnO2 system the Hubbard energy is as a function of the atom type? Are we doing a sort of average of the Hubbard correction on every atom of the same type? \\
\textbf{Answer:} no, in my case the Hubbard potential is only applied to Manganese (Mn1 and Mn2) while for oxygen the orbitals are not localized enough in order for the Hubbard correction to have a sensible effect - this has also been shown in literature.

\paragraph{Q2} What is the constraint in equation (12) of "Hubbard interactions from density functional perturbation theory"? Is it that the number of electrons in the system must be the same? Is it like a normalization of the electronic density?

\paragraph{Q3} What are strongly localized electrons? These are electrons whose orbital is not spread. When these are valence electrons (e.g. in d,f momentum) they need to be described as a wave function thus will particularly suffer from self-interaction error, and get their correlation energy underestimated.
